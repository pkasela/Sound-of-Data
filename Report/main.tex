%%%%%%%%%%%%%%%%%%%%%%%%%%%%%%%%%%%%%%%%%
% Wenneker Article
% LaTeX Template
% Version 2.0 (28/2/17)
%
% This template was downloaded from:
% http://www.LaTeXTemplates.com
%
% Authors:
% Vel (vel@LaTeXTemplates.com)
% Frits Wenneker
%
% License:
% CC BY-NC-SA 3.0 (http://creativecommons.org/licenses/by-nc-sa/3.0/)
%
%%%%%%%%%%%%%%%%%%%%%%%%%%%%%%%%%%%%%%%%%

%----------------------------------------------------------------------------------------
%	PACKAGES AND OTHER DOCUMENT CONFIGURATIONS
%----------------------------------------------------------------------------------------

\documentclass[12pt, a4paper, twocolumn]{article} % 10pt font size (11 and 12 also possible), A4 paper (letterpaper for US letter) and two column layout (remove for one column)

\usepackage[utf8]{inputenc}
\usepackage[italian]{babel}

%%%%%%%%%%%%%%%%%%%%%%%%%%%%%%%%%%%%%%%%%
% Wenneker Article
% Structure Specification File
% Version 1.0 (28/2/17)
%
% This file originates from:
% http://www.LaTeXTemplates.com
%
% Authors:
% Frits Wenneker
% Vel (vel@LaTeXTemplates.com)
%
% License:
% CC BY-NC-SA 3.0 (http://creativecommons.org/licenses/by-nc-sa/3.0/)
%
%%%%%%%%%%%%%%%%%%%%%%%%%%%%%%%%%%%%%%%%%

%----------------------------------------------------------------------------------------
%	PACKAGES AND OTHER DOCUMENT CONFIGURATIONS
%----------------------------------------------------------------------------------------

\usepackage[italian]{babel} % English language hyphenation

\usepackage{microtype} % Better typography

\usepackage{amsmath,amsfonts,amsthm} % Math packages for equations

\usepackage[svgnames]{xcolor} % Enabling colors by their 'svgnames'

\usepackage[hang, small, labelfont=bf, up, textfont=it]{caption} % Custom captions under/above tables and figures

\usepackage{booktabs} % Horizontal rules in tables

\usepackage{lastpage} % Used to determine the number of pages in the document (for "Page X of Total")

\usepackage{graphicx} % Required for adding images

\usepackage{enumitem} % Required for customising lists
\setlist{noitemsep} % Remove spacing between bullet/numbered list elements

\usepackage{sectsty} % Enables custom section titles
\allsectionsfont{\usefont{OT1}{phv}{b}{n}} % Change the font of all section commands (Helvetica)

%----------------------------------------------------------------------------------------
%	MARGINS AND SPACING
%----------------------------------------------------------------------------------------

\usepackage{geometry} % Required for adjusting page dimensions

\geometry{
	top=1cm, % Top margin
	bottom=1.5cm, % Bottom margin
	left=2cm, % Left margin
	right=2cm, % Right margin
	includehead, % Include space for a header
	includefoot, % Include space for a footer
	%showframe, % Uncomment to show how the type block is set on the page
}

\setlength{\columnsep}{7mm} % Column separation width

%----------------------------------------------------------------------------------------
%	FONTS
%----------------------------------------------------------------------------------------

\usepackage[T1]{fontenc} % Output font encoding for international characters
\usepackage[utf8]{inputenc} % Required for inputting international characters

\usepackage{XCharter} % Use the XCharter font

%----------------------------------------------------------------------------------------
%	HEADERS AND FOOTERS
%----------------------------------------------------------------------------------------

\usepackage{fancyhdr} % Needed to define custom headers/footers
\pagestyle{fancy} % Enables the custom headers/footers

\renewcommand{\headrulewidth}{0.0pt} % No header rule
\renewcommand{\footrulewidth}{0.4pt} % Thin footer rule

\renewcommand{\sectionmark}[1]{\markboth{#1}{}} % Removes the section number from the header when \leftmark is used

%\nouppercase\leftmark % Add this to one of the lines below if you want a section title in the header/footer

% Headers
\lhead{} % Left header
\chead{\textit{\thetitle}} % Center header - currently printing the article title
\rhead{} % Right header

% Footers
\lfoot{} % Left footer
\cfoot{} % Center footer
\rfoot{\footnotesize Page \thepage\ of \pageref{LastPage}} % Right footer, "Page 1 of 2"

\fancypagestyle{firstpage}{ % Page style for the first page with the title
	\fancyhf{}
	\renewcommand{\footrulewidth}{0pt} % Suppress footer rule
}

%----------------------------------------------------------------------------------------
%	TITLE SECTION
%----------------------------------------------------------------------------------------

\newcommand{\authorstyle}[1]{{\large\usefont{OT1}{phv}{b}{n}\color{DarkRed}#1}} % Authors style (Helvetica)

\newcommand{\institution}[1]{{\footnotesize\usefont{OT1}{phv}{m}{sl}\color{Black}#1}} % Institutions style (Helvetica)

\usepackage{titling} % Allows custom title configuration

\newcommand{\HorRule}{\color{DarkGoldenrod}\rule{\linewidth}{1pt}} % Defines the gold horizontal rule around the title

\pretitle{
	\vspace{-30pt} % Move the entire title section up
	\HorRule\vspace{10pt} % Horizontal rule before the title
	\fontsize{32}{36}\usefont{OT1}{phv}{b}{n}\selectfont % Helvetica
	\color{DarkRed} % Text colour for the title and author(s)
}

\posttitle{\par\vskip 15pt} % Whitespace under the title

\preauthor{} % Anything that will appear before \author is printed

\postauthor{ % Anything that will appear after \author is printed
	\vspace{10pt} % Space before the rule
	\par\HorRule % Horizontal rule after the title
	\vspace{20pt} % Space after the title section
}

%----------------------------------------------------------------------------------------
%	ABSTRACT
%----------------------------------------------------------------------------------------

\usepackage{lettrine} % Package to accentuate the first letter of the text (lettrine)
\usepackage{fix-cm}	% Fixes the height of the lettrine

\newcommand{\initial}[1]{ % Defines the command and style for the lettrine
	\lettrine[lines=3,findent=4pt,nindent=0pt]{% Lettrine takes up 3 lines, the text to the right of it is indented 4pt and further indenting of lines 2+ is stopped
		\color{DarkGoldenrod}% Lettrine colour
		{#1}% The letter
	}{}%
}

\usepackage{xstring} % Required for string manipulation

\newcommand{\lettrineabstract}[1]{
	\StrLeft{#1}{1}[\firstletter] % Capture the first letter of the abstract for the lettrine
	\initial{\firstletter}\textbf{\StrGobbleLeft{#1}{1}} % Print the abstract with the first letter as a lettrine and the rest in bold
}

%----------------------------------------------------------------------------------------
%	BIBLIOGRAPHY
%----------------------------------------------------------------------------------------

% \usepackage[backend=bibtex,style=authoryear,natbib=true]{biblatex} % Use the bibtex backend with the authoryear citation style (which resembles APA)

\addbibresource{example.bib} % The filename of the bibliography

\usepackage[autostyle=true]{csquotes} % Required to generate language-dependent quotes in the bibliography
 % Specifies the document structure and loads requires packages

% \usepackage{textcomp}
%----------------------------------------------------------------------------------------
%	ARTICLE INFORMATION
%----------------------------------------------------------------------------------------

\title{Sound of Data} % The article title

\author{
  \authorstyle{
    % DO NOT TOUCH HSPACEs!! @Momo
    Riccardo Cervero \hspace{25pt} 000000 \\  % x + 4
    Marco Ferrario \hspace{44pt}  000000 \\   % x + 23
    Pranav Kasela \hspace{46.5pt} 000000 \\   % x + 25.5
    Federico Moiraghi \hspace{21pt} 799735    % x
  } % Authors
  \newline\newline
  \institution{Università degli Studi di Milano Bicocca}
}

\date{Anno Accademico 2018/19} % Add a date here if you would like one to appear underneath the title block, use \today for the current date, leave empty for no date

%----------------------------------------------------------------------------------------

\begin{document}

\maketitle % Print the title

\thispagestyle{firstpage} % Apply the page style for the first page (no headers and footers)

%----------------------------------------------------------------------------------------
%	ABSTRACT
%----------------------------------------------------------------------------------------

\lettrineabstract{Obiettivo del progetto è analizzare la discussione mediatica riguardante i soggetti del mondo musicale contemporaneo e passato.
}
\hfill
\newpage

\tableofcontents
\hfill
\newpage

%----------------------------------------------------------------------------------------
%	ARTICLE CONTENTS
%----------------------------------------------------------------------------------------
\part{Introduzione}
Per raggiungere i traguardi posti dal progetto \textit{Sound of Data}\footnote{\url{https://github.com/pkasela/Sound-of-Data}}, si son dovuti raccogliere dati sufficienti per costruire un \textit{knowledge graph} adeguato alla materia: scaricato un \textit{dump} di \url{musicbrainz.org} come database relazionale (già esportato in formato \textit{Tabular Separated Values} dai manutentori), si è dapprima importato in Apache Hadoop\footnote{\url{https://hadoop.apache.org}} per effettuare una rapida pulizia preliminare e infine esportato in modo tale da costruire un grafo Neo4J\footnote{\url{https://neo4j.com}}.
Costruita quindi la base di conoscenza su cui operare, si sono raccolti \textit{tweet} in tempo reale grazie ad Apache Kafka\footnote{\url{https://kafka.apache.org}}, per poi analizzarli in automatico con un rudimentale strumento di \textit{instance matching}.

\hfill
\newpage
\part{Costruzione dello \textit{knowledge graph}}

\section{Data Cleaning con python e Pig}
I dati vengono scaricati dal sito di musicbrainz attraverso il file get\_data.py, che oltre a scaricare i dati effettua un'analisi preliminare di sostituire tutti i caratteri "\\N" con campi vuoti in tutti i file attraverso il comando sed e salva solo i file necessari per il database a grafo che si vuole creare. 
La lista dei generi è presente sempre sul sito di musicbrainz, con uno scraper viene salvato nel codice e confrontato con il file tag e vengono eliminati tutti i tag che non sono dei generi.

%%%%%%
%%%%%%%%%%%
%%%%%%%%%%%%%%%%
%%%% Ma lo schema iniziale va messo?
%%%%%%%%%%%%%%%%
%%%%%%%%%%%
%%%%%%

I file vengono trasferiti su HDFS per essere processati, in particolare usiamo Apache Pig, che è una piattaforma per processare i dati, la decisione di utilizzare Pig piuttosto che SQL è dovuta al fatto che il caricamento dei dati in SQL è molto lenta e inoltre Pig riesce a sfruttare la potenza di HDFS.
L'engine usato con Pig è Tez, che rende molto più veloce l'esecuzione del processamento dei dati saltando diverse fasi di scrittura dei risultati parziali su HDFS.
%%%%%%
%%%%%%%%%%%
%%%%%%%%%%%%%%%%
%%%% Dopo cercare e mettere il funzionamento di Tez
%%%% Non vorrei scrivere cavolate quindi dopo cerco bene
%%%%%%%%%%%%%%%%
%%%%%%%%%%%
%%%%%%

Una volta finita la pulizia i dati vengono trasferiti fuori da HDFS sul filesystem, e vengono unite i risultati di pig con il comando cat. A questo punto i dati sono pronti per essere caricati su Neo4j.

%%%%%%
%%%%%%%%%%%
%%%%%%%%%%%%%%%%
%%%% Conviene mettere lo schema finale delle tabelle (dopo lo metto)
%%%%%%%%%%%%%%%%
%%%%%%%%%%%
%%%%%%

\section{Neo4J}

I dati puliti da Pig vengono caricati su neo4j usando neo4j-import.
%cosa altro si deve scrivere hahaha

\hfill
\newpage
\part{Analisi dei tweet}
\section{Analisi con Apache Kafka}
\section{Riconoscimento delle istanze nel testo}
Data la mole di tweet scaricabili, si è deciso di costruire un strumento di \textit{instance matching} creato \textit{ad hoc} per i tweet.
Vista la scarsa abilità di algoritmi basati su reti neurali e \textit{deep learning} ad analizzare il breve (e quindi decontestualizzato) testo di un tweet, si è costruito un modello per l'identificazione di ipotetiche entità su cui basarsi per confronti col database (grazie alla API\footnote{\url{https://python-musicbrainzngs.readthedocs.io/en/v0.6/}} offerta da \url{musicbrainz.org} stesso).

\subsection{Identificazione delle entità}
Le entità sono riconosciute non mediante \textit{machine learning} ma grazie a semplici stratagemmi linguistici.
Prima di tutto il testo del tweet è ripulito da eventuali abbreviazioni gergali; subito dopo sono ricercate le parole nel testo che non risultano essere italiane: analizzando tweet in lingua italiana, si presume che una stringa in lingua diversa abbia una certa importanza.
Per fare questo è usato un mero correttore ortografico che evidenzia quali parole non sono riconosciute; di aiuto nel compito è anche una semplice espressione regolare che tenta di stabilire quali parole non seguono la costruzione sillabica italiana (rientrando quindi o nella categoria dei sostantivi della quinta classe o, nei casi fortunati, nelle entità cercate): secondo le regole linguistiche, una sillaba correttamente formata è composta da un numero massimo di tre consonanti seguita da una vocale e da al più una sola consonante (o vocale con suono consonantico, formando quindi un dittongo).
Alle entità così individuate si aggiungono tutte le parole scritte in maiuscolo (che in un testo di così bassa formalità non sempre coincidono coi nomi propri), anche se a inizio frase.
Grazie all'uso della punteggiatura (le virgolette) e delle preposizioni, si tenta inoltre di stabilire se l'entità rilevata è un presunto autore o una presunta opera e quindi cercata all'interno del database.

\subsubsection{Prestazioni del modello}

\subsubsection{Margini di miglioramento}

\subsection{Riconoscimento delle entità}

\hfill
\newpage
\part{Visualizzazioni dei dati}

\hfill
\newpage
\part{Risultati e conclusioni}



%----------------------------------------------------------------------------------------
%	BIBLIOGRAPHY
%----------------------------------------------------------------------------------------

\hfill
\newpage
\printbibliography[title={Bibliografia}]

%----------------------------------------------------------------------------------------

\end{document}
